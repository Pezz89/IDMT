\documentclass[titlepage]{scrartcl}
\usepackage{enumitem}
\usepackage[british]{babel}
\usepackage[style=apa, backend=biber]{biblatex}
\DeclareLanguageMapping{british}{british-apa}
\usepackage{url}
\usepackage{float}
\restylefloat{table}
\usepackage{perpage}
\MakePerPage{footnote}
\usepackage{abstract}
\usepackage{graphicx}
% Create hyperlinks in bibliography
\usepackage{hyperref}

\usepackage[T1]{fontenc}
\usepackage[utf8]{inputenc}
\setkomafont{disposition}{\normalfont\bfseries}

\usepackage{blindtext}
\setkomafont{disposition}{\normalfont\fontsize{12}{17}\bfseries}
\setkomafont{section}{\normalfont\fontsize{12}{17}\bfseries}
\setkomafont{subsection}{\normalfont\fontsize{12}{17}\itshape}
\setkomafont{subsubsection}{\normalfont\fontsize{12}{17}\itshape}

\graphicspath{{./resources/}}
\addbibresource{~/Documents/library.bib}

\usepackage{etoolbox}
\makeatletter
\makeatother

\DeclareCiteCommand{\citeyearpar}
    {}
    {\mkbibparens{\bibhyperref{\printdate}}}
    {\multicitedelim}
    {}


\begin{document}
    \title{ECS742\\Interactive Digital Media Techniques\\Assignment 2: Processing Game}
    \subtitle{\LARGE{Report}}
    \author{Sam Perry\\Student Number: 160842984}
    \date{}

    \maketitle


    \section*{Procedure}
    Following instructions provided to create the initial game was a relatively
    straight forward task, given previous experience in programming. The most
    prominent problem was an array indexing error that occured when the
    projectile left the screen. This was quickly fixed by adding conditions to
    check if the projectile was on screen before attempting to access the
    groundLevel array.

    On completion of this, a number of further features were implemented to
    build on the basic game. These included adding graphics to the background
    (and other aesthetic changes), generating more detailed maps at random, and
    adding a blast radius to projectiles when coliding with objects.

    \section*{Findings}
    Project illustrated the cross-over between a number of animation and DSP
    techniques: Terrain generation, framerate and collision detection.
    
    The main difficulty, although still rather minor, lied in adjusting to the
    java language. As familiar concepts such as passing variables by reference
    do not exist in the language, learning the ``java/processing way'' of
    implementing features was the main skill aquired in this project.

    There are many further improvments that could be implemented in this
    project. However, it would be of great benefit to refactor a large amount
    of the code to take advantage of programming models such as OOP. This would
    allow for quicker and less ``hacky'' implementations of features and would
    improve overall code quality.
    \printbibliography
\end{document}
